\documentclass[letterpaper,12pt]{article}
%documentclass[superscriptaddress,preprintnumbers,amsmath,amssymb,aps,11pt]{revtex4}
%\usepackage[]{authblk}
%\usepackage{graphics}
\usepackage[dvipdf]{graphics}
%\usepackage{subfig}  % For subfloats
\usepackage{color}
\usepackage[usenames,dvipsnames]{xcolor}
\usepackage{epsfig}
\usepackage{wrapfig}
\usepackage{rotating}
\usepackage{caption}
%\usepackage{subcaption}
\usepackage{subfig}
\usepackage{authblk}
\usepackage{url}

\oddsidemargin = -14mm
\topmargin = -2.9cm
\textwidth = 19cm
\textheight = 24cm

\def \rarr {\rightarrow}
\def \grinp {\includegraphics}
\def \tw {\textwidth}
\def\dfrac#1#2{\displaystyle{{#1}\over{#2}}}
\def \dstl {\displaystyle}
\definecolor{GREEN}{rgb}{0.,0.8,0}
\definecolor{RED}{rgb}{1,0,0}
\definecolor{ORANGE}{rgb}{1,0.5,0}

\title{Validation of EC trigger using GEMC}
\date \today
\author{Rafyayel Paremuzyan}

\begin{document}
 \maketitle
 To Validate EC Stage1 trigger, we have generated events that contain electron, in this case a reaction $\gamma p \rightarrow e^{-}e^{+}p$ was used, then this events
 were passed through GEMC, and from GEMC knowing the actual kinematics (coordinate, energy and momentum) of electrons and positrons just before entering the Electromagnetic 
 Calorimeter (EC), we looked into the EC trigger cluster banks to see if there exist a cluster close (by coordinate and energy) to the GEMC particle.
 NOTE: All the detectors except EC were removed from the simulations to avoid creation multiple secondaries from the interaction of the original particle with other detectors.6
 
 Steps are following: For each particle from GEMC that entered EC volume, it was checked, whether 
 \begin{enumerate}
  \item \label{enum:Steps_Trig_check} There is an EC trigger bank in that event
  \item \label{enum:Steps_Sect_check}If yes, then, whether there is a trigger bank for that sector
  \item \label{enum:Steps_Clust_check}If then, check if there is cluster(s) in that sector
  \item \label{enum:Steps_dthdphi_check}If yes, then compare Trigger cluster(s) coordinates to the particle coordinates from the GEMC.
 \end{enumerate}

 In the Fig.\ref{fig:ECYXC_fails} shown ``$Y$ vs $X$" coordinate distributions of particle coordinates from the GEMC, that failed certain conditions in the List mentioned above.
 (a) to (d) correspond conditions 1 to 4 in the list above correspondingly.
%%%%%%%%%%%%%%%%%%%%%%%%%%%%%%%%%%%%%%%%%%%%%%%%%%%%%%%% F I G U R E %%%%%%%%%%%%%%%%%%%%%%%%%%%%%%%%%%%%%%%%%%%%%%%%%%%%%%%%%
 \begin{figure}[!htb]
 \centering
 \subfloat[]{\label{fig:ECTrue_yxc_Trig_check}\grinp[width=0.45\tw]{../figs/ECTrue_yxc1.png}}
 \subfloat[]{\label{fig:ECTrue_yxc_Sect_check}\grinp[width=0.45\tw]{../figs/ECTrue_yxc2.png}}\\
 \subfloat[]{\label{fig:ECTrue_yxc_Clust_check}\grinp[width=0.45\tw]{../figs/ECTrue_yxc3.png}}
 \subfloat[]{\label{fig:ECTrue_yxc_dthdphi_check}\grinp[width=0.45\tw]{../figs/ECTrue_yxc4.png}}\\
 \caption{``$Y$ vs $X$" coordinate distributions of particle coordinates from the GEMC, that failed certain conditions in the List mentioned above.}
 \label{fig:ECYXC_fails}
 \end{figure}
%%%%%%%%%%%%%%%%%%%%%%%%%%%%%%%%%%%%%%%%%%%%%%%%%%%%%%%% F I G U R E %%%%%%%%%%%%%%%%%%%%%%%%%%%%%%%%%%%%%%%%%%%%%%%%%%%%%%%%%
As one can see there are significant amount of events, when there is a VTP trigger bank in the corresponding sector, but no clusters are observed.
These events also show distinct pattern on the EC face. In total these events are about $10\%$ of all GEMC particles.
%%%%%%%%%%%%%%%%%%%%%%%%%%%%%%%%%%%%%%%%%%%%%%%%%%%%%%%% F I G U R E %%%%%%%%%%%%%%%%%%%%%%%%%%%%%%%%%%%%%%%%%%%%%%%%%%%%%%%%%
 \begin{figure}[!htb]
  \centering
  \subfloat[]{\label{fig:dth1}\grinp[width=0.45\tw]{../figs/dth1.png}}
  \subfloat[]{\label{fig:dPhi1}\grinp[width=0.45\tw]{../figs/d_Ph1.png}}\\
  \subfloat[]{\label{fig:Proj_dth1}\grinp[width=0.45\tw]{../figs/Proj_dTh1.png}}
  \subfloat[]{\label{fig:Proj_dPhi1}\grinp[width=0.45\tw]{../figs/Proj_dPhi1.png}}
  \caption{$\Delta \theta$ (left) and $\Delta \Phi$ (right) distributions as a function of sector number (top row) and integrated over all sectors (bottom). Red lines indicate cut limits on a corresponding variable.}
  \label{fig:dthdphi}
 \end{figure}
%%%%%%%%%%%%%%%%%%%%%%%%%%%%%%%%%%%%%%%%%%%%%%%%%%%%%%%% F I G U R E %%%%%%%%%%%%%%%%%%%%%%%%%%%%%%%%%%%%%%%%%%%%%%%%%%%%%%%%%

Now let's look into matching of triggered clusters and GEMC particles. In Fig.\ref{fig:dthdphi} shown 
$\Delta \theta$ (left) and $\Delta \Phi$ (right) distributions as a function of sector number (top row) and integrated over all sectors (bottom). Red lines indicate cut limits on a corresponding variable.
This tells that trigger clusters are quite close to GEMC hits, and imposed cuts remove only tiny (less than $1\%$) fraction of GEMC events.

%%%%%%%%%%%%%%%%%%%%%%%%%%%%%%%%%%%%%%%%%%%%%%%%%%%%%%%% F I G U R E %%%%%%%%%%%%%%%%%%%%%%%%%%%%%%%%%%%%%%%%%%%%%%%%%%%%%%%%%
\begin{figure}
 \centering
 \subfloat[]{\label{fig:EC_E_Noclust}\grinp[width=0.45\tw]{../figs/EC_E3.png}}
 \subfloat[]{\label{fig:ECYXcTrueMatched}\grinp[width=0.45\tw]{../figs/ECTrue_yxc_Mathced.png}}
\end{figure}
%%%%%%%%%%%%%%%%%%%%%%%%%%%%%%%%%%%%%%%%%%%%%%%%%%%%%%%% F I G U R E %%%%%%%%%%%%%%%%%%%%%%%%%%%%%%%%%%%%%%%%%%%%%%%%%%%%%%%%%

\newpage
\section{Energy in sector 5}
Looking energy distributions of clusters, seems Energy of clusters in sector 5 is quite different from other sector energies.
In Fig.\ref{fig:TrigClust_E} Left shown Sector vs cluster Energy (in ADC units)). Middle: Energy distribution in sector 1. RIght: Energy distribution in sector 5.
%%%%%%%%%%%%%%%%%%%%%%%%%%%%%%%%%%%%%%%%%%%%%%%%%%%%%%%% F I G U R E %%%%%%%%%%%%%%%%%%%%%%%%%%%%%%%%%%%%%%%%%%%%%%%%%%%%%%%%%
\begin{figure}
 \centering
 \subfloat[]{\label{fig:TrigClustE_vsSect}\grinp[width=0.3\tw]{../figs/EC_cl_E.png}}
 \subfloat[]{\label{fig:TrigClustE_sec1}\grinp[width=0.3\tw]{../figs/Proj_ECEcl_sec1.png}}
 \subfloat[]{\label{fig:TrigClustE_sec5}\grinp[width=0.3\tw]{../figs/Proj_ECEcl_sec5.png}}
 \caption{Left: Sector vs cluster Energy (in ADC units). Middle: Energy distribution in sector 1. RIght: Energy distribution in sector 5.}
 \label{fig:TrigClust_E}
\end{figure}
%%%%%%%%%%%%%%%%%%%%%%%%%%%%%%%%%%%%%%%%%%%%%%%%%%%%%%%% F I G U R E %%%%%%%%%%%%%%%%%%%%%%%%%%%%%%%%%%%%%%%%%%%%%%%%%%%%%%%%%

\newpage
\section{Overflow in U, V and W peaks}
In Fig.\ref{fig:EC_peak_overflows} shown U (left), V (middle) and W (right) peak coordinate distributions. As one can see, there are few instances of overflow events.
%%%%%%%%%%%%%%%%%%%%%%%%%%%%%%%%%%%%%%%%%%%%%%%%%%%%%%% F I G U R E %%%%%%%%%%%%%%%%%%%%%%%%%%%%%%%%%%%%%%%%%%%%%%%%%%%%%%%%%
\begin{figure}[!htb]
\centering
 \subfloat[]{\grinp[width=0.32\tw]{../figs/EC_coord_U_peak.png}}
 \subfloat[]{\grinp[width=0.32\tw]{../figs/EC_coord_V_peak.png}}
 \subfloat[]{\grinp[width=0.32\tw]{../figs/EC_coord_W_peak.png}}\\
\caption{U (left), V (middle) and W (right) peak coordinate distributions. there are few instances of overflow events.}
\label{fig:EC_peak_overflows}
\end{figure}
%%%%%%%%%%%%%%%%%%%%%%%%%%%%%%%%%%%%%%%%%%%%%%%%%%%%%%%% F I G U R E %%%%%%%%%%%%%%%%%%%%%%%%%%%%%%%%%%%%%%%%%%%%%%%%%%%%%%%%%

\newpage
\section{Summary of questions/issues}
\vskip 1cm
About $10\%$ of electrons/positrons entering into EC, doesn't provide clusters. These events show distinct pattern on EC face.
\vskip 1cm \noindent
Energy of clusters in Sector 5 is quite different from other sector energies
\vskip 1cm \noindent
There are some overflows in EC coordinate distributions
\vskip 2cm \noindent
During the studying of matching trigger clusters and GEMC hits, I had to reverse order of readig V and W strips, in order to be able match  Trigger clusters to GEMC, otherwise they don't match. 
This needs to be discussed and sorted out.

\end{document}
